% Created 2018-06-26 Tue 13:24
% Intended LaTeX compiler: xelatex
\documentclass[11pt]{article}
\usepackage{graphicx}
\usepackage{grffile}
\usepackage{longtable}
\usepackage{wrapfig}
\usepackage{rotating}
\usepackage[normalem]{ulem}
\usepackage{amsmath}
\usepackage{textcomp}
\usepackage{amssymb}
\usepackage{capt-of}
\usepackage{hyperref}
\author{IOANNIS ZANNOS}
\date{\today}
\title{Magnetic Dance\\\medskip
\large An Art-Science Experiment in Embodied Interaction}
\hypersetup{
 pdfauthor={IOANNIS ZANNOS},
 pdftitle={Magnetic Dance},
 pdfkeywords={},
 pdfsubject={},
 pdfcreator={Emacs 27.0.50 (Org mode 9.1.7)}, 
 pdflang={English}}
\begin{document}

\maketitle
\tableofcontents


\section{Introduction}
\label{sec:orgc20f3e9}

\subsection{Overview}
\label{sec:org5162467}
The title of this presentation could be "Dancing and Weaving Songs with Electrons and Quanta:  Emergence of metric patterns in telematic dance performance."  It relates a project that was conceived in January 2018, just 5 months before the start of my residency in Tokyo, and is still in its beginning phase.  The project is ambitious because of the need that presented itself to bridge the distance between Japan and Europe through simultaneous live performance, in through telematic dance in several cities, Tokyo, Athens, Brussels and Manchester.  The impulse for this conception was circumstantial, as I wanted to participate in an exhibition that is taking place right now in Athens because of the affinity of its topic to my ongoing research direction in Art-Science, and the only clearly worthwhile way to do this seemed to be through a live performance that did not just consist of me sitting behind my laptop in Tokyo and sending data online to Athens.  Thus began a task whose complexity exploded rapidly at an alarming rate, as did its theoretical ramifications and interest for me.

The project spans a wide spectrum of scientific and artistic fields, from Artificial Life and Quantum Biology to Telematic Dance, Live Coding and Embodied Interaction in Music Performance.  Furthermore, it combines several techniques and tools for live performance, including wearable sensor devices, live-coding systems for music performance and telematic performance using the internet.  As such, the technical challenges proved to be far greater than anticipated, and progress was delayed, as new know-how had to be acquired in addition to developing novel, untested approaches, mostly relying on low-cost open-source hardware and software.  There is thus little to show at this point, but I am greatful to Ricardo Climent for his support and the offer to present at this workshop, despite the risks.  Therefore, I will try here to convey a sense of the journey so far. I will start from the project's conception and introduce the disparate ideas that entered it in the manner of actor appearances in a play.  Then I will outline the technical development work so far and give some examples.  If time permits, a taste of the development and technical work will be given through sample demo sessions of communication tests and programming of wearable embedded devices. Finally, I will mention some plans for the immediate future, and open up the stage for discussion.  

\subsection{Background}
\label{sec:orgcc201df}

The project arose under the circumstances of my sabbatical research stay in Tokyo in Spring and Summer of 2018.  Due to the need to participate at an Art-Science exhibition organised by my colleague Prof. Spyros Verykios during that period, I had to devise a way to perform remotely that would be original and make sense in the context of the exhibition. As it developed, the project incorporated other research topics and ideas that had preoccupied me since my participation in the previous installment of the EASTN Project (2014-2015) and in the preparatory phase of EASTN-DC, of which this workship is part.  It is not clear yet how much and in what manner these concepts will be applied to the project in practice.  Rather, these are questions that drive the design of future experiments and research within the project.  It is as such that the ideas below as presented, as an account of the ideas driving the project in order to give an insight to the mechanisms underlying its process, and not as a scientific exposé.

\subsubsection{Initial stimulus: From sonification of magnetic fields to biosemiotics}
\label{sec:orgdbe2a54}

The initial prompt for project came from an Art-Science exhibition organized by my colleague at the Audiovisual Arts Department of Ionian University, Prof. Spyros Verykios.  Inspired by the 1966 Science Fiction film "Fantastic Voyage", Spyros formulated a call for works dealing with navigating the inner space of humans both figuratively and literally, and by extension also the discovery of parallels between Inner and Outer Space (in the thelematic sense, "As Above So Bewow").  The exhibition is showing at the National Observatory of Athens (NOA) till end of this month (\url{https://cosmonauts1.wixsite.com/cosmonauts/athens}).  I had been invetigating the possibility of a collaboration with NOA dealing with the sonification of magnetic storms since 2017, and thus I thought it would be a good opportunity to participate.  However, since I had planned to be on sabbatical in Tokyo from May 2018, the only way to participate with a live event would be remotely. The initial idea was to scan the local magnetic field of the two locations involved, Athens and Tokyo using easily available magnetic sensors such as those used in mobile devices for navigation, and to use the data for sonification.  The performers would wear magnetometers on their body and explore variations in the ambiend magnetic field in this way, usind sound as feedback for their orientation.  Already this basic idea seem interesting as it introduced magnetism as a sense source for humans, which suggested parallels to alternative sense forms in other life forms, such as microorganisms or plants.  Just as these organism use phototropism or chemotropism to adapt to search for energy sources in their environment, the dancers would use the ambient magnetic field to navigate or adapt their sonic environment as a metaphoric source of energy.  Moreover the use of such an exotic stimulus source as sense suggested associations with early work in the field of biosemiotics which had interested me for a while.  Jakob von Uexküll's 1934 paper "A Stroll through the Worlds of Animals and Men: A Picture Book of Inisible Worlds" proposed a very early model of perception of the environment built in bottom up manner - without the preconceptions about perception inherent in human-centric views. In this paper, he introduced the notion of environment (Umwelt) in Biology and Comparative Psychology, and thus opened the way to Environmental Sciences.  As the title suggests, the article argues that each biological entities world view is built by using the signals received from the sensory organs at its disposal, and that therefore different animals can have radically different views or perceptions of the world.  By studying the perception and behavior mechanism of simple animals such as the tick, a parasite that uses only the senses of smell and temperature to find its prey, Uexküll opened the way to cybernetics.  Furthermore, his suggestion that there is not just one valid world view, but many views dependent on the sense and action mechanisms of the perceiving subjects, prefigures some basic tenets of Radical Constructivism developed amongst others by biologists Umberto Maturana and Francisco Varela.  


\subsubsection{Enter Artificial Life and Autopoiesis}
\label{sec:org9ba7f98}

\subsubsection{Enter Quantum Entanglement}
\label{sec:org30fd9ce}

\subsubsection{Enter Live Coding, Weaving, Embodied Interaction and Traditional African Music}
\label{sec:orgfa89587}

\subsubsection{Enter Soundscape Composition, Rhythmanalysis and the Niche Hypothesis}
\label{sec:org80a0972}

The question arises at this stage, how can experiments in telematic dance with simple wearable sensor devices relate to in practice or contribute to research along the ideas 

\section{Implementation}
\label{sec:orga62cbe2}

\subsection{Sensors}
\label{sec:orgb51caa0}

\subsection{Wearable Wireless Technology}
\label{sec:org4cb8101}

\subsection{Software for Embedded Devices and Internet of Things (IoT)}
\label{sec:orgc6d6bb9}

\subsection{Peerformance Model on SuperCollider}
\label{sec:orgb797068}

\subsection{Network aspects}
\label{sec:orga006ce5}

\subsection{Partners}
\label{sec:org06b39ad}

\section{Outlook}
\label{sec:orge52c774}

\section{References}
\label{sec:org94ed9b4}



Uexküll, Jakob von. 1934.  "A Stroll through the Worlds of Animals and Men: A Picture Book of Invisible Worlds" (original Title: "Streifzüge durch die Umwelten von Tieren und Menschen: Ein Bilderbuch unsichtbarer Welten"). In: Verständliche Wissenschaft, Vol. 21. Berlin, J. Springer.

Virgo, Nathaniel. 2011. Thermodynamics and the Structure of Living Systems. PhD Thesis, University of Sussex.

Virgo, Nathaniel, \ldots{}. 2011. 


\subsection{Web}
\label{sec:org08b9ed3}
\url{http://sro.sussex.ac.uk/6334/1/Virgo\%2C\_Nathaniel.pdf}


\url{http://www.massey.ac.nz/\~wwpapajl/evolution/lecture1/docs/tick.htm}
\end{document}
